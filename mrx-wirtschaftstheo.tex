\documentclass[10pt,a4paper, ngerman]{beamer}
%%% https://docs.google.com/document/d/1SgbuYOFK2A63enCco90Hdt2dBV0MToXd4Bq2IoN8glU/edit#
\include{beamer-with-code}

\AtBeginSection{\frame{\frametitle{Gliederung}\tableofcontents[currentsection]}}

\setbeamercovered{transparent}
\author{Luca Hartmann \and Luca Kiebel \and Luca Hülsmann}
\title{Marxistische Wirtschaftstheorie}
%\subtitle{subtitle}
\date{\today}
\institute[HBBK]{Hans-Böckler-Berufskolleg}
\setlength{\itemsep}{10pt}
\begin{document}
\begin{frame}
\titlepage
\end{frame}

\section{Wert- und Geldtheorie}
\begin{frame}
  \frametitle{Wert- und Geldtheorie}
  Karl Marx differenziert Zwei Themen im Handelsbereich:
  \pause
  \begin{itemize}
    \item Wert einer Ware
    \pause
    \item Was ist Geld?
  \end{itemize}
\end{frame}
\subsection{Wert einer Ware}
\begin{frame}
  \frametitle{Wert einer Ware}
  Wie bestimmt sich der Wert einer Ware?
  \pause
  \begin{itemize}
    \item Gebrauchswert
    \pause
    \begin{itemize}
      \item Die Nützlichkeit der Ware bestimmt den Gebrauchswert
    \end{itemize}
    \pause
    \item Tauschwert
    \pause
    \begin{itemize}
      \item Der Aufwand zur Produktion bestimmt den Tauschwert
    \end{itemize}
  \end{itemize}
\end{frame}
\subsection{Was ist Geld?}
\begin{frame}
  \frametitle{Was ist Geld?}
  Marx unterteilte Geld in Vier Aspekte
  \pause
  \begin{itemize}
    \item Geld als Selbstständige Wertform \pause
    \item Geld als Kapital \pause
    \item Geld als Maßstab der Preise \pause
    \item Geld als Ausdruck des Wertmaßes
  \end{itemize}
\end{frame}

\section{Krisentheorie}
\begin{frame}
  \begin{itemize}
    \item Wirtschaftliche Entwicklung führte zu immer niedrigerem Niveau 
    \pause
    \item Kapital kann schneller vermehrt werden, als die Anzahl der Arbeitskräfte wächst 
    \pause
    \item mehr Kapital pro Arbeiter, aber kein proportionaler Mehrverdienst
    \pause
    \item Risiko einer Investition ist zu hoch um sie zu wagen
    \pause
    \item Investitionsstreik stürzt die Wirtschaft in eine Krise
\end{frame}

\begin{frame}
  Carl Christian von Weizsäcker aus Bonn und Lawrence Summer (damaliger Premierminister der USA) 
  argumentierten, dass die Menschheit soviel investiert hatte und es keine rentablen
  Investitionsprodukte übrig bleiben.
\end{frame}

\section{Das Kapital}
\subsection{Die Zusammensetzung des Kapitals}
\subsection{Die Reproduktion des Kapitals}


\section{Kritik der politischen Ökonomie}
\begin{frame}
  \begin{itemize}
    \item von Karl Marx von August 1858 bis Januar 1859 verfasst
    \pause 
    \item Wert von Waren als Gebrauchswert und Tauschwert 
    \pause
    \item konkrete und abstrakte Arbeit
    \pause 
    \item Arbeitszeit als quantitatives Maß für Arbeit
  \end{itemize}
\end{frame}

\section{Preisarten}



\end{document}
